% scrap_am
% https://docs.google.com/document/d/1K74-eiI-uxx01WNpQJL8RVL8zVeoj2bfZC3cEuD2pWM/edit
% !TEX encoding = UTF-8 Unicode
%\documentclass[12pt]{article}%\documentclass{article}
\documentclass[parskip=half,fontsize=12pt]{scrartcl}%\documentclass[parskip=full]{scrartcl}
\usepackage[T1]{fontenc}
\usepackage[utf8]{inputenc}
\usepackage{amssymb,amsmath,amsthm} 
\usepackage[papersize={400pt,520pt},margin=15pt]{geometry}% e-readers
%\usepackage{geometry}% normal
%\usepackage[margin=15pt]{geometry} old
%\usepackage[papersize={4.5in,6in},margin=0.5cm]{geometry} old
%\usepackage[parfill]{parskip}%https://tex.stackexchange.com/questions/133003/using-the-parskip-package-i-find-the-space-between-subtitles-ugly
\usepackage[pdfusetitle]{hyperref}
\usepackage{datetime}
\usepackage[osf]{Baskervaldx} % oldstyle figures
%\usepackage{Baskervaldx}
\usepackage[baskervaldx]{newtxmath}
\usepackage{tikz-cd}
\usepackage{comment}
\pagestyle{empty}
%\newcommand{\nn}{\newcommand}
\newcommand{\oo}{\operatorname}
\newcommand{\mf}{\mathfrak}
\newcommand{\aaa}{\mf a}
\newcommand{\bbb}{\mf b}
\newcommand{\ppp}{\mf p}
\newcommand{\bu}{\bullet}
\newcommand{\ds}{\displaystyle}
\newcommand{\epi}{\twoheadrightarrow}
\newcommand{\incl}{\hookrightarrow}
\newcommand{\mono}{\rightarrowtail}

%\addtolength{\parskip}{.2\baselineskip}
\newtheorem{thm}{Theorem}[section]
\newtheorem{rk}[thm]{Remark}
\newtheorem{qn}[thm]{Question}
\title{About Atiyah and MacDonald's Book}
\author{Pierre-Yves Gaillard}
\date{\today,\currenttime}

\begin{document}%\ar[r,yshift=4pt]\ar[yshift=0.7ex]{r}\ar[yshift=-0.7ex]{r}\large%{\oo c}

\section{Page 67, Exercise 5.2}\label{67}%

Set $\mf p:=\oo{Ker}f$ and let $\mf q\subset B$ be given by Theorem 5.10 p.~62. Our problem can be summarized as follows: 
$$
\begin{tikzcd}
A/\mf p\ar[r,hook]\ar[d,hook]&B/\mf q\ar[dl,dashrightarrow,hook]\\ 
\Omega.
\end{tikzcd}
$$ 
Writing $K$ and $L$ for the respective fields of fractions of $A/\mf p$ and $B/\mf q$, we get the problem 
$$
\begin{tikzcd}
K\ar[r,hook]\ar[d,hook]&L\ar[dl,dashrightarrow,hook]\\ 
\Omega.
\end{tikzcd}
$$ 
As $L/K$ is algebraic and $\Omega$ algebraically closed, this problem has a solution.

\end{document}
